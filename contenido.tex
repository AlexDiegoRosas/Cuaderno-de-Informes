\section{Introducción}

\noindent En esta sección se coloca la introducción a la temática abordada por primera vez en el abstract. Introduce los conceptos a tratar en un enfoque teórico y señala los elementos que serán empleados para el desarrollo experimental del reporte o trabajo. Nota que este párrafo no está indentado. Si quieres quitar el indentado manualmente, con noindent puedes hacerlo.

\section{Metodología de la práctica}

En esta sección se pueden introducir la forma de trabajo que se empleará para realizar las actividades experimentales a realizar en la práctica. Si no requieres esta sección, puedes eliminarla, o bien, modificarla para que se amolde a tus necesidades.

Si necesitas ecuaciones en una línea separada sin numerar, puedes usar esto:

$$s_f=s_0+v_0t+\frac{1}{2}at^2$$

o las puedes escribir en línea de esta forma: $v_f^2=v_i^2+2a\Delta s$ o $\xi$.

\section{Experimentos}

Aquí puedes colocar cada uno de los experimentos que realizaste para la elaboración de tu práctica, trabajo, o proyecto. Aquí tienes un ejemplo de como colocar una imagen. No te tienes que preocupar por enumerarlas, ¡LaTeX ya lo hace por ti! Las puedes citar en cualquier parte como [\ref{exemploLabel}].

\begin{figure}[h!]
    \centering
    %Modificando los parámetros de height puedes cambiar el tamaño de tu imagen.
    \includegraphics[height=9cm]{Imágenes/DummyExp.jpg}
    \caption{\textit{Simón profe, si le entendimos a la práctica.}}
    \label{exemploLabel}
    \end{figure}

\section{Resultados}

Aquí puedes poner todos los resultados que obtuviste, incluyendo tablas, gráficos, dibujos o esquemas. Aquí abajo tienes como ejemplo una tabla con LaTeX. Puedes crearlas con generadores online, o si se te complica, lo más fácil es hacerlas en Excel y solo capturar la imagen para incluirla como el ejemplo anterior. 

\begin{table}[ht]
\begin{center}
\caption{Todas las tablas necesitan un pie de foto en la parte superior. Fijate en la consistencia de los valores de las tablas. No olvides tus unidades.}
\label{table1} 
\begin{tabular}{ccc} %change to cc for 2 columns
\hline
\multicolumn{1}{c}{Distance, $d$ (km) } & \multicolumn{1}{c}{Voltage, $V\ (\pm 0.05$ V)} & \multicolumn{1}{c}{Current, $I$\ (mA $\pm 5$\%)}\\
\hline
1.2 $\pm$ 0.2 &  0.30 & 20 \\
1.6 $\pm$ 0.4 &  0.21 & 30 \\
2.5 $\pm$ 0.1 &  0.18 & 40 \\
5.9 $\pm$ 0.2 &  0.13 & 50 \\
\hline
\end{tabular}
\end{center}
\end{table}


\section{Conclusion}
Finalmente, aquí puedes poner tus conclusiones, sean individuales o por equipo. Si son por equipo, una forma fácil de hacerlo es con una lista itemize:

\begin{itemize}
  \item \textbf{Nombre 1:} si me gustó la práctica jsjs.

  \item \textbf{Nombre 1:} no me gustó la práctica :(

\end{itemize}
