\section*{\centering\textbf{INSTRUCCIONES PARA EL USO DEL CUADERNO DE INFORMES}}
\vspace{1cm}
\section{\normalsize{ PRESENTACIÓN.}}

El cuaderno de informes es un documento de autocontrol, en el cual el estudiante,
registra diariamente, durante la semana, las tareas, operaciones que ejecuta en su
aprendisaje, es un medio para desarrollar la Competencia de Redactar Informes. 

\section{\normalsize{INSTRUCCIONES PARA EL USO DEL CUADERNO DE INFORMES}}
\begin{itemize}
\item {En la hoja de informe semanal, el estudiante registrará los trabajos
que ejecuta, indicando el tiempo correspondiente. El día de asistencia 
registraá los contenidos que desarrolla. Al término de la semana totalizará
las horas.\\
De las tareas ejecutadas durante la semana, el ESTUDIANTE seleccionará la 
tarea más significativa y él hará una descripción del proceso de ejecución 
con esquemas, diagramas y dibujos correspondientes que aclaren dico proceso.}

\item {Semanalmente, el instructor revisará y calificará el Cuaderno de Informes
haciendo las observaciones y recomendaciones que considere convenientes, en los 
aspectos relacionados a la elaboracion de un Informe Técnico (letra normalizada,
dibujo técnico, descripción de la tarea y su procedimiento, normas técnicas,
seguridad, etc.)}

\item {Escala de calificación vigesimal}
\end {itemize}

\vspace{2cm}

\begin{table}[!ht]
  \centering
  \begin{tabular}{| c | c | c |}
    \hline
    \rowcolor{gray!30}
    \textbf{CUANTITATIVA} & \textbf{CUALITATIVA} & \textbf{CONDICIÓN} \\ \hline
    $16.8-20.0$ & Excelente & \multirow{3}{*}{Aprobado} \\ \cline{1-2}
    $13.7-16.7$ & Bueno &  \\ \cline{1-2}
    $10.5-13.6$ & Aceptable &  \\ \hline
    $00-10.4$   & Deficiente & Desaprobado \\ \hline
  \end{tabular}
  \caption{Cuadro de calificaciones}
  \label{tab:calificaciones}
\end{table}

